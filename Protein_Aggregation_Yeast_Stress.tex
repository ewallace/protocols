\documentclass{article}

% amsmath package, useful for mathematical formulas
\usepackage{amsmath}
% amssymb package, useful for mathematical symbols
\usepackage{amssymb}
% gensymb package, useful for general symbols, such as degrees celsius
\usepackage{gensymb}
% appropriate spaces after macro insertions
\usepackage{xspace} 

% graphicx package, useful for including eps and pdf graphics
% include graphics with the command \includegraphics
\usepackage{graphicx}
\usepackage{rotating}
\usepackage[margin=2cm]{geometry}
\usepackage{color}

% enumitem allows continuous numbering across split lists.
\usepackage{enumitem}

%% EW macros
\newcommand{\mul}{\ensuremath{\mu}L\xspace}
\newcommand{\mug}{\ensuremath{\mu}g\xspace}
\newcommand{\mum}{\ensuremath{\mu}m\xspace}
\newcommand{\degC}{\celsius\xspace}
\newcommand{\tb}[1]{\textcolor{blue}{#1}}
%\newcounter{PExCounter}

\begin{document}
\title{\vspace{-.75in} Measurement of protein aggregation by fractionating yeast}
\author{Edward Wallace, ewjwallace@gmail.com}
\date{\today}
\maketitle

%To do: 
%\begin{itemize}
%  \item Figures for all gels.
%  \item GeneElute spin column?
%\end{itemize}

%\section*{Introduction}
%
%The aim is to find out which RNAs are sequestered in stress granules, by extracting both soluble and insoluble portions of lysed cells, with and without heat shock, and performing RNASeq on both fractions. The insoluble fraction of heat-shocked cells is enriched for stress granules. The observation that small subunit of ribosomal proteins are found in stress granules suggests that we should see more 18S rRNA (from the small subunit) than 28S rRNA (from the large) in stress-granule samples. The basic idea is to lyse cells, separate the lysate into soluble and insoluble fractions, and extract RNA from each fraction.

This protocol is for harvesting of yeast and separation of their proteins into aggregate/membrane and soluble fractions, for further analysis by SDS-PAGE, western blot,  mass spectrometry, etc. Based on Mike Dion and Allan Drummond's 2011 protocols, with modifications by Edward Wallace\footnote{Previous versions of this file were called solubleInsolubleProtein*.*} and Quincey Justman. 

\subsection*{Prepare}

%Prepare: 
\begin{itemize}
  \item soluble protein buffer (SPB; 20mM HEPES-NaOH pH7.4, 120mM KCl, 2mM EDTA, 0.2mM DTT, 1:100 PMSF, 1:100 protease inhibitors cocktail IV). Make stock of salt and buffer, add DTT and inhibitors shortly before use, and chill.
\item total protein buffer (TPB;   20mM HEPES-NaOH pH7.4, 150mM NaCl, 5mM EDTA, 3\% SDS, 1:100 PMSF, 2mM DTT, 1:1000 protease inhibitors IV). 
  \item insoluble protein buffer (IPB; 8M Urea, 20mM HEPES-NaOH pH7.4, 150mM NaCl, 2\% SDS, 2mM EDTA, 2mM DTT, 1:100 PMSF, 1:1000 protease inhibitors IV.). Mix fresh daily the urea, DTT, and inhibitors, and keep at room temperature:
  
%\begin{itemize}[itemsep=0pt,leftmargin=40pt]
%  \item[0.96g\ ] urea
%  \item[818\mul] H2O
%  \item[200\mul] 20\% SDS (w/v)
%  \item[\ 40\mul] 1M HEPES-NaOH pH7.4
%  \item[\ 20\mul] 100mM PMSF
%  \item[\ 20\mul] 200mM DTT
%  \item[\ \ 2\mul] protease inhibitors
%\end{itemize}
\begin{tabular}{rrl}
for 2mL & for 13mL & \\
950\mul & 6.2mL & H2O \\
75\mul &  390\mul & 5M NaCl \\
8\mul &  52\mul & 0.5M EDTA \\
200\mul & 1.3mL & 20\% SDS (w/v) \\
\ 40\mul & 260\mul & 1M HEPES-NaOH pH7.4 \\
\multicolumn{3}{l}{\tb{Or make IPB stock from above ingredients, and use:}} \\
\tb{1.27 mL} & \tb{8.2mL} & \tb{IPB Stock} \\
  0.96g &  6.24g & urea \\
\ 20\mul & 130\mul & 100mM PMSF \\
\ 4\mul & 26\mul & 1M DTT \\
\ \ 2\mul & 13\mul & protease inhibitors \\
\end{tabular}

  Use 500\mul IPB per sample, and note that Urea is slow to dissolve, and foaming will make $\sim$25\% of the solution unusable. IPB solidifies at 4\degC due to urea and SDS; also, don't substitute KCl as it precipitates SDS even at RT.

  %  \item 
%    \item For heat or stress treatment, prepare media in advance and pre-warm.
     \item For heat treatment, pre-warm water bath to 46\degC or pre-warm media as required.
      \item Appropriate numbers of safe-lok tubes loaded with 7mm steel balls, racked in liquid nitrogen (LN2).
    \item Pre-chill a centrifuge, also label and lay out all tubes and equipment in advance as the protocol moves very quickly once started. 
    \item Protocol can be paused and samples stored at -80\degC when cells have been flash-frozen (step \ref{step:flashfreeze}), and after cells have been ground in mixer mill (step \ref{step:powder}).
%    \item less sample could be used, but these quantities are easy to handle 
\end{itemize}


\subsection*{Sample Growth}
\begin{enumerate}[resume]
%Grow up several cultures of yeast (BY4741) to $4 \times 10^6$ cells/mL ($OD_{600} \approx 0.45$) at 30\degC, in 100ml media in a 250ml flask. Pretreat cells as desired. For
\item Grow up a culture of Saccharomyces cerevisiae (BY4741) to $4 \times 10^6$ cells/mL ($OD_{600} \approx 0.45$) at 30\degC, with 120rpm shaking, in 300ml SC-complete medium in a 1000ml erlenmeyer flask.
%\item For heat shock, prepare 2x 100mL of SC-complete medium in a 250ml flask and pre-warm at 42\degC. For glucose withdrawal, prepare 2x100mL of SC-glc medium, substituting 2\% sorbitol for glucose, in a 250ml flask and pre-warm at 30\degC. Prepare a control 2x 100ml of SC-complete at 30\degC.
%\item Set up a  cellulose filter (Millipore 1.2\mum RAWP09025) on a cleaned Kontes glass filtration apparatus. Wet filter with prewarmed medium.
%\item For each treatment, pour 100ml of cell culture onto filter and drain under mild house vacuum; culture should take a few seconds to drain through and should never completely dry out. Wash immediately with 100ml treatment media. With tweezers, roll filter paper into almost a cylinder, insert longways into flask of 100ml treatment media, and agitate to resuspend yeast from filter. Place at appropriate incubation temperature with shaking.
% \item At the desired treatment time (5 minutes or 60 minutes), harvest the cells as below. Take control after 5mins of incubation (length of shortest perturbation treatment), and stage all treatment times as closely as possible: set up heat-shock and take 5 min timepoint, then set up -glc and take 5min timepoint, then set up and take control timepoint, then longer timepoints.
\end{enumerate}

\subsection*{Sample Lysis}


\begin{enumerate}[resume]
\item Transfer $2 \times 10^8$ cells to a 50mL conical tube (50 mL of a $4 \times 10^6$ cells/mL culture). 
\item Spin at 2500g for 30s in a swinging bucket rotor at RT. The end of this spin marks the start of the timed treatment duration. Gently decant and discard supernatant. For heat shock treatment, hold tube containing pellet in waterbath at desired temperature for desired time; alternatively, transfer to pre-warmed media for desired time and end by spinning as described.
%\item Spin at 2500g for 30s in a swinging bucket rotor at 4\degC. The start of this spin marks the end of the timed treatment duration. Gently decant and discard supernatant~\footnote{Alternately, for short heat shock treatment in pellets, spin down cells at RT discard supernatant, then hold tube containing pellet in waterbath at desired temperature for desired time. }.
%\item Optional: 

\item Resuspend pellet in 1ml ice-cold SPB, on ice, and transfer to 1.5ml tube.
\item Spin at 5,000g, 4\degC, for 30 seconds.
\item Resuspend new pellet in 150\mul lysis buffer.
\item \label{step:flashfreeze} 
Drip 100uL of resuspended pellet onto upper wall of tube containing steel ball, still racked in LN2. Goal is to get a nugget of frozen material on the wall, and to avoid dripping the material around the ball and thus freezing the ball to the bottom of the tube; having some LN2 remaining in the tube helps.  Place the remaining 100uL of resuspended pellet in a tube for total protein extraction: process, or freeze, immediately. 
\item Place tubes at -80\degC; when all LN2 has boiled out of tube (listen -- if any popping or hissing, keep waiting), snap tube closed carefully, away from other tubes. Keep in LN2. (Any remaining LN2 in tube will cause tube to explode open and fire the stainless steel ball into your iPad, brain, colleague, or other important equipment.)
\item Rack the tube into the PTFE 2mL tube adaptor for the Retsch Mixer Mill MM400 (Retsch \#22.008.0005) and submerge the entire assembly in LN2.
Agitate for $4\times 90$ seconds at 30 Hz in a Retsch Mixer Mill MM400, returning sample holder to LN2 between sessions. Complete lysis produces fine snowy powder in the tube.
\item  \label{step:powder} 
Remove sample tubes from LN2, tap on bench to release powder from lid, and pop the caps to relieve pressure. 
\item Add 400 \mul SPB to each tube, thaw on ice with occasional vortexing, and as soon as possible extract ball with a magnet. (We rinse balls in methanol and store in 50\% ethanol.)
\end{enumerate}

\subsection*{Soluble fraction extraction}
\begin{enumerate}[resume]
  \item Spin at 3000g for 30 seconds (clarification step) to remove whole cells and very large aggregates.
  \item Decant clarified liquid into a 1.5mL microcentrifuge tube. If desired, keep the pellet and process it alongside the insoluble fraction; this end product is the \emph{unclarified fraction}.
  \item Spin at 100,000g for 20 minutes (fixed-angle TLA-55 rotor at 40,309 rpm, 4\degC, in a Beckman Coulter Optimax tabletop ultracentrifuge).
  \item Decant supernatant into a 1.5mL microcentrifuge tube: this is the \emph{soluble fraction}. 
  \item Take 10ul aliquot of soluble fraction and mix with Laemmli buffer; use this to run a protein gel and assess protein integrity.
\end{enumerate}

\subsection*{Insoluble fraction extraction}
\begin{enumerate}[resume]
  \item Violently snap pellet to clear remaining liquid.
  \item Add 500 \mul soluble protein buffer (SPB) and vortex violently. (The pellet may not resuspend; that's fine.)
  \item Spin at 100,000g for 20 minutes.
  \item Discard supernatant, clear residual liquid with a hard snap.
  \item Add 250 \mul insoluble protein buffer (IPB); note this means insoluble sample is relatively 2X concentrated to others. Process samples in IPB at room temperature to maintain solubility of the Urea.
  \item Dislodge the pellet with a pipet tip, Vortex until pellet dissolves, 10-15 minutes for clarified samples.
  \item Spin at 20,000g, RT, for 5 minutes.
  \item Decant supernatant into a 1.5mL microcentrifuge tube: this is the \emph{insoluble fraction}.
  \item Run a 4-15\% SDS-page gel; load roughly 5-10\mul of total and soluble fractions and 2-4 times the quantity of insoluble fractions. Make aliquots for further analysis.
\end{enumerate}

\subsection*{Total protein extraction}
\begin{enumerate}[resume]
\item Add 400 \mul TPB to each total protein tube. Incubate at 95�C with 500rpm mixing for 20 min.
\item Vortex vigorously for 15 min.
\item Spin at 6000g, RT, 1 min. Take supernatant; this is the \emph{total protein fraction}. 
\end{enumerate}

\subsection*{Chloroform:Methanol Extraction}
For Mass Spectrometry runs, or shipping proteins at room temperature, first perform a precipitation. Detergents, like SDS, and salts, like NaCl, can disrupt LC-MS/MS runs. Precipitation with chloroform and methanol results in dry protein material, free of salt and detergent. Adapted from Wessel, D. and Flugge, U.I. (1984) Anal. Biochem. 138 141--143.

\begin{enumerate}[resume]
\item To 100\mul protein sample (~100\mug protein) in a 1.5mL tube:
\item Add 400\mul methanol and vortex thoroughly.
\item Add 100\mul chloroform and vortex.
\item Add 300\mul H2O---mixture will become cloudy with precipitate---and vortex.
\item Centrifuge 1 minute at 14,000g. Result is three layers: a large aqueous layer on top, a circular flake of protein in the interphase, and a smaller chloroform layer at the bottom.
\item Remove top aqueous layer carefully, trying not to disturb the protein flake.
\item Add 400\mul methanol and vortex.
\item Centrifuge 5 minutes at 20,000g, which will slam dandruffy precipitate against the tube wall.
\item Remove as much methanol as possible. Be careful, because the pellet is delicate. You should be able to remove all but a few \mul of methanol with care, which will speed drying.
\item Dry under vacuum, and seal tube. This may be stored at RT short-term or -20\degC long term, prior to resuspension.
\end{enumerate}





\end{document}